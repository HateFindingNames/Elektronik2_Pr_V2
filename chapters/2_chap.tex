%ltex: language=de
\chapter{Simulation}
	\section{Ausgangsspannung}
		Zu ermitteln ist einerseits der Peak des Überschwingens im Einschaltmoment der Schaltung nach \cref{fig:basisschaltung}.
		Das Datenblatt gibt unter Kapitel 7.6 eine Ausgangsspannung von \(\SI{4,8}{V} \leq U_{out} \leq \SI{5,2}{V}\) bei einer Eingangsspannung von
		zwischen \SI{7}{V} und \SI{40}{V}. Hier wurde der minimale, aus den Testbedingungen des Datenblattes zu entnehmende Laststrom von \SI{2}{mA}
		gewählt.
		\begin{figure}[h]
			\centering
			% \includegraphics[width=.8\textwidth]{LTSpice/figures/einschaltpeak.png}
			\includesvg[width=.8\textwidth]{LTSpice/2.1/basis_schematic_LM2596_peak}
			\caption[Spannungsspitze nach dem Einschalten]{Nach dem Einschalten steigt die Ausgangsspannung auf \(\approx \SI{6,5908}{V}\)
			um sich nach etwa \SI{2,5}{ms} eingeschwungen zu haben.}
			\label{fig:einschaltpeak}
		\end{figure}
		%
		Um die mittlere Spannung im eingeschwungenen Zustand zu ermitteln, wurden die Simulationsanweisung dahingehend geändert, dass
		nur Werte ab \SI{5}{ms} nach dem Einschalten aufgezeichnet werden. Daraus gibt das Simulationsprogramm eine mittlere
		Spannung am Ausgang von \SI{4,9886}{V} an, was gut im vom Hersteller angegebenen Intervall liegt.
		%
	\newpage
	\section{Restwelligkeit}
		Ohne die Schaltung aus \cref{fig:basisschaltung} zu verändern wurde die Restwelligkeit der Ausgangsspannung bei
		Lastströmen von \SI{0,2}{A} und bei \SI{3}{A} ermittelt. Hierzu wurde der Spannungsverlauf zwischen \SI{5}{ms} und \SI{10}{ms}
		nach dem Einschaltmoment aufgetragen. \Cref{fig:ripple} zeigt einen Zoom auf das Zeitintervall \SI{(2-2,1)}{ms}.
		Der Verlauf in Grün für einen Laststrom von \SI{0,2}{A} zeigt einen Spitze-Spitze-Wert von \(U_{pp} = \SI{0,05106}{V}\).
		Die Ausgangsspannung vür einen Laststrom von \SI{3}{A} ist in Blau eingezeichnet. Hier ist ein Spitze-Spitze-Wert von \(U_{pp} = \SI{0,05891}{V}\)
		abzulesen.
		\begin{figure}[h]
			\centering
			% \includegraphics[width=.8\textwidth]{LTSpice/2.2/ripple_wf.jpg}
			\includesvg[width=.8\textwidth]{LTSpice/2.2/restwelligkeit}
			\caption[Restwelligkeit der Ausgangsspannung.]{Restwelligkeit der Ausgangsspannung bei \SI{0,2}{A} (solid) und \SI{3}{A} (gepunktet).}
			\label{fig:ripple}
		\end{figure}
		%
	\section{Schaltfrequenz}
		Es soll die Periodendauer des pulsweitenmodulierten Signals am \textsf{Out}-Anschluss des \textit{LM2596} bei \SI{12}{V} Eingangsspannung ermittelt werden.
		Wie in \cref{fig:schaltfrequenz 12V waveform} exemplarisch wurde hierzu die Cursor-Funktion im Waveform-Diagram der Simulation benutzt.
		An zwei aufeinanderfolgenden steigenden Flanken des Ausgangssignals wurde jeweils ein Cursor angelegt und die entsprechenden Zeiten
		\(t_2\) und \(t_1\) im zugehörigen Fenster abgelesen.
		\begin{figure}[h]
			\centering
			\includesvg[width=.8\textwidth]{LTSpice/2.3/periodendauer_12V}
			\caption[Ermittlung der Periodendauer des Ausgangssignals des \textit{LM2596}]{Ermittlung der Periodendauer des Ausgangssignals des \textit{LM2596} durch \(t_2 - t_1\) mit jeweils einem
			Cursor an zwei aufeinander folgenden steigenden Flanken des Signals.}
			\label{fig:schaltfrequenz 12V waveform}
		\end{figure}
		Hier ergeben sich eine Periodendauer und Frequenz des Oszillators zu
		\begin{gather}
			T = t_2 - t_1 = \SI{2,01}{ms} - \SI{2,0033}{ms} = \SI{6,7}{\micro s} \nonumber \\
			\Leftrightarrow \nonumber \\
			f = \frac{1}{T} \approx \SI{149,25}{kHz}
		\end{gather}
		Der Hersteller gibt hierzu einen typischen Wert von \(\SI{(150 \pm 23}{kHz}\) an (siehe \cref{fig:auszug kapitel 7.9}). Der in der Simulation
		ermittelte Wert liegt ausreichend nah am erwarteten Wert. Da es sich bei der Diskrepanz um \(< \SI{0,1}{\micro s}\) handelt, lässt sich der Unterschied
		durch Ungenauigkeiten bei der Platzierung der Cursor erklären.\par\medskip
		Der Tastgrad des PWM-Signals ergibt sich mit
		\begin{equation}
			DC = \frac{t_{on}}{T} \cdot \SI{100}{\percent}
		\end{equation}
		zu etwa \SI{43}{\percent}.\par
		Um zu untersuchen, wie sich eine Erhöhung der Eingangsspannung auf die Parameter Tastgrad und Periodendauer auswirken, wurde die gleiche Simulation
		mit Eingangsspannungen von \SI{20}{V} und \SI{30}{V} durgchgeführt. Die Ergebnisse werden in \cref{tab:schaltfrequenzen} aufgeführt.
		\begin{table}[h]
			\centering
			\caption[Tabelle mit für die Simulationsergebnisse bezüglich der Periodendauer und des Tastgrades]{Tabelle mit für die Simulationsergebnisse bezüglich der Periodendauer und des Tastgrades in Abhängigkeit der Höhe der Eingangsspannung.}
			\begin{tabular}{@{}llllll@{}}
				\toprule
							&$t_1$				&$t_2$				&$T$				&$t_{on}$			&$DC$	\\
				\midrule
				\SI{12}{V}	&\SI{2,0033}{ms}	&\SI{2,0100}{ms}	&\SI{6,7}{\micro s}	&\SI{2,9}{\micro s}	&\SI{43}{\percent}	\\
				\SI{20}{V}	&\SI{2,0043}{ms}	&\SI{2,0110}{ms}	&\SI{6,7}{\micro s}	&\SI{1,9}{\micro s}	&\SI{28}{\percent}	\\
				\SI{30}{V}	&\SI{2,0034}{ms}	&\SI{2,0100}{ms}	&\SI{6,6}{\micro s}	&\SI{1,2}{\micro s}	&\SI{18}{\percent} \\
				\bottomrule
			\end{tabular}
			\label{tab:schaltfrequenzen}
		\end{table}
		%
	\section{Spulenstrom}
		Untersucht werden soll das Verhalten des Spulenstroms bei Änderung der Last und dem damit einhergehenden Laststrom.
		\begin{figure}[h]
			\centering
			\includesvg[width=.8\textwidth]{LTSpice/2.4/spulenstrom}
			\caption[Zoom auf den Verlauf des Spulenstroms \(I_L\)]{Zoom auf den Verlauf des Spulenstroms \(I_L\) (grün) und der pulsweitenmodulierten Ausgangsspannung (blau) bei einem Laststrom von \SI{3}{A}. Die Welligkeit \(\Delta I_L\) beträgt
			\SI{590,69}{mA} bei einem RMS von \SI{3,0053}{A} zwischen \SI{5}{ms} und \SI{10}{ms} nach dem Einschalten.}
			\label{fig:spulenstrom}
		\end{figure}%
		Es wird eine Simulation mit der unveränderten Schaltung wie in \cref{fig:basisschaltung} dargestellt durchgeführt und der Spulenstrom
		im Intervall von \SI{5}{ms} bis \SI{10}{ms} nach dem Einschalten aufgetragen.
		\Cref{fig:spulenstrom} zeigt einen Zoom auf den
		Zeitbereich zwischen \SI{7}{ms} und \SI{7,025}{ms}. Der Abstand zwischen den minimalen und maximalen Spitzenwerten entlang der Vertikalen 
		gibt die Welligkeit \(\Delta I_L\) des Spulenstroms. Hier beträgt sie \SI{590,69}{mA} mit einem RMS im angegebenen Zeitintervall
		von \SI{3,0053}{A}.\par
		Reduktion des Laststromes auf \SI{1}{A} erhöht die Welligkeit des Spulenstroms auf \SI{603,91}{mA}. Der RMS beträgt hier im oben genannten Intervall
		\SI{1,0156}{A}.\par\medskip
		Weitere Reduktion des Laststroms wirkt sich kaum auf die Welligkeit des Spulenstroms aus. Allerdings kann sein absoluter Wert
		so weit sinken, dass sein minimale Spitzenwert \SI{0}{A} erreicht. Simulationen zeigen, dass diese Situation bei \(\approx \SI{0.3}{A}\)
		erreicht ist.\par
		Wenn während des Betriebs Momentanwerte von \SI{0}{A} im Verlauf Spulenstroms existieren, wird dies \textit{Discontinuous Current Mode} genannt.
		In diesen Bereichen können parasitäre Kapazitäten der Diode zusammen mit der Induktivität der Spule einen Schwingkreis erzeugen, dessen
		Spitzenwerte in ungünstigen Fällen die Eingangsspannung übersteigen und so Teile der Schaltung zerstören können. Kapitel 8.4.1 des
		Datenblattes folgend, kann dem entgegnet werden, indem einerseits die Induktivität der Spule entsprechend angepasst wird und/oder
		indem mittels eines der Spule parallel geschalteten Widerstandes ungewollte Schwingungen gedämpft werden. In jedem Fall jedoch
		sollte die Schaltung entsprechnd der erwarteten Lastströme dimensioniert werden.
		%
	\section{Maximaler Ausgangsstrom}
		Das Datenblatt zum \textit{LM2596} gibt in Kapitel 7.9 unter \textit{Current Limit} einen maximalen Ausgangsstrom (typisch) von
		\SI{4,5}{A} an (vgl. \cref{fig:auszug kapitel 7.9}). Um dies zu untersuchen wird die Schaltung mit \SI{0,1}{\ohm} belastet.
		Rechnerisch wäre hier bei einer Ausgangsspannung von \(\approx \SI{5}{V}\) ein Strom von \(\approx \SI{50}{A}\) zu erwarten.
		\begin{figure}[h]
			\centering
			\includesvg[width=.8\textwidth]{LTSpice/2.5/max_ausgangsstrom}
			\caption[Maximaler Ausgangsstrom]{Verlauf des Ausgangsstroms der Schaltung und sein Maximalwert bei einer Last von \SI{0,1}{\ohm}.}
			\label{fig:max ausgangsstrom}
		\end{figure}
		Die Simulation ergibt hier einen maximalen Wert von \SI{4,933}{A} bei einem RMS Wert zwischen \SI{5}{ms} und \SI{10}{ms} nach dem Einschalten von \SI{4.8594}{A}.
		\Cref{fig:max ausgangsstrom} stellt den Verlauf des Stromes durch den Lastwiderstand dar. Zu Erkennen ist eine deutliche Verzerrung des Verlaufs
		bei einer Überlagerung einer sekundärschwingung auf die Ausgangsspannung. Gleichzeitiges Auftragen des PWM-Signals zeigt, dass weder die Periodendauer des
		internen Oszillators, noch der Tastgrad des PWM-Signals konstant sind.
		%
	\section{Wirkungsgrad des Netzteils}
		Zur Beurteilung des Wirkungsgrades der Schaltung in \cref{fig:basisschaltung} wird die Simulation angewiesen, die abgegebene Leistung der Spannungsquelle, sowie
		die aufgenommene Leistung der Last aufzutragen. Um Vergleichbarkeit zu gewährleisten, sollen die Simulationsparameter deckungsgleich mit den Testbedingungen des Datenblattes
		unter Kapitel 7.6 (vgl. \cref{fig:auszug kapitel 7.6}) sein. Differenz der Mittelwerte beider Leistungen ergibt nach \cref{eq:wirkungsgrad} einen Wirkungsgrad von \(\approx \SI{90,7}{\percent}\).
		\begin{align}
			\eta = \frac{\bar{P}_out}{\bar{P}_in} \cdot \SI{100}{\percent} = \frac{\SI{14,963}{W}}{\SI{16,485}{W}} \cdot \SI{100}{\percent} \approx \SI{90,7}{\percent}
			\label{eq:wirkungsgrad}
		\end{align}%
		Der simulierte Wert liegt deutlich über der Angabe des Datenblattes mit \SI{80}{\percent}.